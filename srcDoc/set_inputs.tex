This sets the starting time of the simulation.
Even when you restart, the starttime should be
to the real start time, not the restart time.
\begin{verbatim}
#STARTTIME
iYear    (integer)
iMonth   (integer)
iDay     (integer)
iHour    (integer)
iMinute  (integer)
iSecond  (integer)
\end{verbatim}

This sets the ending time of the simulation.
\begin{verbatim}
#ENDTIME
iYear    (integer)
iMonth   (integer)
iDay     (integer)
iHour    (integer)
iMinute  (integer)
iSecond  (integer)
\end{verbatim}

This will set a time for the code to just pause.
Really, this should never be used.
\begin{verbatim}
#PAUSETIME
iYear iMonth iDay iHour iMinute iSecond
\end{verbatim}

This is typically only specified in a
restart header.  If you specify it in a start UAM.in
it will start the counter to whatever you specify.
\begin{verbatim}
#ISTEP
iStep     (integer)
\end{verbatim}

This sets the maximum CPU time that the code should
run before it starts to write a restart file and end
the simulation.  It is very useful on systems that
have a queueing system and has limited time runs.
Typically, set it for a couple of minutes short of
the max wall clock, since it needs some time to write
the restart files.
\begin{verbatim}
#CPUTIMEMAX
CPUTimeMax    (real)
\end{verbatim}

If you just want GITM to run MSIS and IRI over and
over again, use this.  Then GITM is never actually
run.  You just get MSIS and IRI.
\begin{verbatim}
#STATISTICALMODELSONLY
UseStatisticalModelsOnly    (logical)
DtStatisticalModels         (real)
\end{verbatim}

This is typically only specified in a
restart header.
It sets the offset from the starttime to the
currenttime. Should really only be used with caution.
\begin{verbatim}
#TSIMULATION
tsimulation    (real)
\end{verbatim}

Sets the F10.7 and 81 day average F10.7.  This is
used to set the initial altitude grid, and drive the
lower boundary conditions.
\begin{verbatim}
#F107
f107  (real)
f107a (real - 81 day average of f107)
\end{verbatim}

This specifies the initial conditions and the
lower boundary conditions.  For Earth, we typically
just use MSIS and IRI for initial conditions.
For other planets, the vertical BCs can be set here.
\begin{verbatim}
#INITIAL
UseMSIS        (logical)
UseIRI         (logical)
If UseMSIS is .false. then :
TempMin        (real, bottom temperature)
TempMax        (real, top initial temperature)
TempHeight     (real, Height of the middle of temp gradient)
TempWidth      (real, Width of the temperature gradient)
\end{verbatim}

This says how to use tides.  The first one is using
MSIS with no tides.  The second is using MSIS with
full up tides. The third is using GSWM tides, while
the forth is for experimentation with using WACCM
tides.
\begin{verbatim}
#TIDES
UseMSISOnly        (logical)
UseMSISTides       (logical)
UseGSWMTides       (logical)
UseWACCMTides      (logical)
\end{verbatim}

If you selected to use GSWM tides above, you
can specify which components to use.
\begin{verbatim}
#GSWMCOMP
GSWMdiurnal(1)        (logical)
GSWMdiurnal(2)        (logical)
GSWMsemidiurnal(1)    (logical)
GSWMsemidiurnal(2)    (logical)
\end{verbatim}

This is probably for damping vertical wind
oscillations that can occur in the lower atmosphere.
\begin{verbatim}
#DAMPING
UseDamping        (logical)
\end{verbatim}

I dont know what this is for...
\begin{verbatim}
#GRAVITYWAVE
UseGravityWave        (logical)
\end{verbatim}

This sets the hemispheric power of the aurora.
Typical it ranges from 1-1000, although 20 is a
nominal, quiet time value.
\begin{verbatim}
#HPI
HemisphericPower  (real)
\end{verbatim}

I dont think that GITM actually uses this unless
the Foster electric field model is used.
\begin{verbatim}
#KP
kp  (real)
\end{verbatim}

The CFL condition sets how close to the maximum time
step that GITM will take.  1.0 is the maximum value.
A value of about 0.75 is typical.  If instabilities
form, a lower value is probably needed.
\begin{verbatim}
#CFL
cfl  (real)
\end{verbatim}

This sets the driving conditions for the high-latitude
electric field models.  This is static for the whole
run, though.  It is better to use the MHD\_INDICES
command to have dynamic driving conditions.
\begin{verbatim}
#SOLARWIND
bx  (real)
by  (real)
bz  (real)
vx  (real)
\end{verbatim}

Use this for dynamic IMF and solar wind conditions.
The exact format of the file is discussed further
in the manual.
\begin{verbatim}
#MHD_INDICES
filename  (string)
\end{verbatim}

This is for using Pat Newells aurora (Ovation).
\begin{verbatim}
#NEWELLAURORA
UseNewellAurora   (logical)
UseNewellAveraged (logical)
UseNewellMono (logical)
UseNewellWave (logical)
UseNewellRemoveSpikes (logical)
UseNewellAverage      (logical)
\end{verbatim}

\begin{verbatim}
#AMIEFILES
cAMIEFileNorth  (string)
cAMIEFileSouth  (string)
\end{verbatim}

The limiter is quite important.  It is a value
between 1.0 and 2.0, with 1.0 being more diffuse and
robust, and 2.0 being less diffuse, but less robust.
\begin{verbatim}
#LIMITER
TypeLimiter  (string)
\end{verbatim}

This will set how much information the code screams
at you - set to 0 to get minimal, set to 10 to get
EVERYTHING.  You can also change which processor is
shouting the information - PE 0 is the first one.
If you set the iDebugLevel to 0, you can set the dt
of the reporting.  If you set it to a big value,
you wont get very many outputs.  If you set it to a
tiny value, you will get a LOT of outputs.
UseBarriers will force the code to sync up a LOT.
\begin{verbatim}
#DEBUG
iDebugLevel (integer)
iDebugProc  (integer)
DtReport    (real)
UseBarriers (logical)
\end{verbatim}

\begin{verbatim}
#THERMO
UseSolarHeating   (logical)
UseJouleHeating   (logical)
UseAuroralHeating (logical)
UseNOCooling      (logical)
UseOCooling       (logical)
UseConduction     (logical)
UseTurbulentCond  (logical)
UseUpdatedTurbulentCond  (logical)
EddyScaling  (real)
\end{verbatim}

\begin{verbatim}
#THERMALDIFFUSION
KappaTemp0    (thermal conductivity, real)
\end{verbatim}

\begin{verbatim}
#VERTICALSOURCES
UseEddyInSolver              (logical)
UseNeutralFrictionInSolver   (logical)
MaximumVerticalVelocity      (real)
\end{verbatim}

\begin{verbatim}
#EDDYVELOCITY
UseBoquehoAndBlelly              (logical)
UseEddyCorrection   (logical)
\end{verbatim}

\begin{verbatim}
#WAVEDRAG
UseStressHeating              (logical)
\end{verbatim}

If you use eddy diffusion, you must specify two pressure
levels - under the first, the eddy diffusion is constant.
Between the first and the second, there is a linear drop-off.
Therefore The first pressure must be larger than the second!
\begin{verbatim}
#DIFFUSION
UseDiffusion (logical)
EddyDiffusionCoef (real)
EddyDiffusionPressure0 (real)
EddyDiffusionPressure1 (real)
\end{verbatim}

\begin{verbatim}
#FORCING
UsePressureGradient (logical)
UseIonDrag          (logical)
UseNeutralFriction  (logical)
UseViscosity        (logical)
UseCoriolis         (logical)
UseGravity          (logical)
\end{verbatim}

\begin{verbatim}
#DYNAMO
UseDynamo              (logical)
DynamoHighLatBoundary  (real)
nItersMax              (integer)
MaxResidual            (V,real)
\end{verbatim}

\begin{verbatim}
#IONFORCING
UseExB                 (logical)
UseIonPressureGradient (logical)
UseIonGravity          (logical)
UseNeutralDrag         (logical)
UseDynamo              (logical)
\end{verbatim}

\begin{verbatim}
#DIPOLE
MagneticPoleRotation   (real)
MagneticPoleTilt       (real)
xDipoleCenter          (real)
yDipoleCenter          (real)
zDipoleCenter          (real)
\end{verbatim}

\begin{verbatim}
#APEX
UseApex (logical)
        Sets whether to use a realistic magnetic
        field (T) or a dipole (F)
\end{verbatim}

\begin{verbatim}
#ALTITUDE
AltMin                (real, km)
AltMax                (real, km)
UseStretchedAltitude  (logical)
\end{verbatim}

If LatStart and LatEnd are set to < -90 and
> 90, respectively, then GITM does a whole
sphere.  If not, it models between the two.
If you want to do 1-D, set nLons=1, nLats=1 in
ModSizeGitm.f90, then recompile, then set LatStart
and LonStart to the point on the Globe you want
to model.
\begin{verbatim}
#GRID
nBlocksLon   (integer)
nBlocksLat   (integer)
LatStart     (real)
LatEnd       (real)
LonStart     (real)
LonEnd       (real)
\end{verbatim}

\begin{verbatim}
#STRETCH
ConcentrationLatitude (real, degrees)
StretchingPercentage  (real, 0-1)
StretchingFactor      (real)
Example (no stretching):
#STRETCH
65.0 ! location of minimum grid spacing
0.0	 ! Amount of stretch 0 (none) to 1 (lots)
1.0  
\end{verbatim}

\begin{verbatim}
#TOPOGRAPHY
UseTopography (logical)
\end{verbatim}

\begin{verbatim}
#RESTART
DoRestart (logical)
\end{verbatim}

This will allow you to change the output cadence
of the files for a limited time.  If you have an event
then you can output much more often during that event.
\begin{verbatim}
#PLOTTIMECHANGE
yyyy mm dd hh mm ss ms (start)
yyyy mm dd hh mm ss ms (end)
\end{verbatim}

For satellite files, you can have one single file
per satellite, instead of one for every output.
This makes GITM output significantly less files.
It only works for satellite files now.
\begin{verbatim}
#APPENDFILES
DoAppendFiles    (logical)
\end{verbatim}

This sets the output files.  The most common type
is 3DALL, which outputs all primary state variables.
Types include : 3DALL, 3DNEU, 3DION, 3DTHM, 3DCHM, 
3DUSR, 3DGLO, 2DGEL, 2DMEL, 2DUSR, 1DALL, 1DGLO, 
1DTHM, 1DNEW, 1DCHM, 1DCMS, 1DUSR.
\begin{verbatim}
#SAVEPLOT
DtRestart (real, seconds)
nOutputTypes  (integer)
Outputtype (string, 3D, 2D, ION, NEUTRAL, ...)
DtPlot    (real, seconds)
\end{verbatim}

\begin{verbatim}
#SATELLITES
nSats     (integer - max = ',nMaxSats,')
SatFile1  (string)
DtPlot1   (real, seconds)
etc...
\end{verbatim}

Sets the time for updating the high-latitude
(and low-latitude) electrodynamic drivers, such as
the potential and the aurora.
\begin{verbatim}
#ELECTRODYNAMICS
DtPotential (real, seconds)
DtAurora    (real, seconds)
\end{verbatim}

\begin{verbatim}
#LTERadiation
DtLTERadiation (real)
\end{verbatim}

This is really highly specific.  You dont want this.
\begin{verbatim}
#IONPRECIPITATION
UseIonPrecipitation     (logical)
IonIonizationFilename   (string)
IonHeatingRateFilename  (string)
\end{verbatim}

You really want a log file.  They are very important.
It is output in UA/data/log*.dat.
You can output the log file at whatever frequency
you would like, but if you set dt to some very small
value, you will get an output every iteration, which
is probably a good thing.
\begin{verbatim}
#LOGFILE
DtLogFile   (real, seconds)
\end{verbatim}

This is for a FISM or some other solar spectrum file.
\begin{verbatim}
#EUV_DATA
UseEUVData            (logical)
cEUVFile              (string)
\end{verbatim}

