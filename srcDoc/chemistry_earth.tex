%  Copyright (C) 2002 Regents of the University of Michigan, portions used with permission 
%  For more information, see http://csem.engin.umich.edu/tools/swmf
\documentclass[twoside,10pt]{book}
\usepackage{times}
\usepackage{graphicx}
\usepackage{alltt}
\usepackage{amsmath}
\usepackage{epsfig}
\usepackage{fancyhdr}
\usepackage[square]{natbib}
\usepackage{multicol}
\usepackage{subfigure}
\usepackage{fancyvrb}
\usepackage{color}

% use these lengths for a more uniform margin
% this format is more pleasing for stapling
\setlength{\oddsidemargin}{-.1 in}
\setlength{\evensidemargin}{0.0 in}
\setlength{\textwidth}{6.5 in}
\setlength{\topmargin}{0 in}
\setlength{\textheight}{8.5 in}

\renewcommand{\deg}{^{\circ}}


\title{GITM Chemical Scheme for Earth  \\ \large Version 2.1}
\author{A.J. Ridley}

\begin{document}

\pagestyle{fancy}
\lhead[\fancyplain{}{\bfseries\thepage}]{\fancyplain{}{\bfseries\rightmark}}
\rhead[\fancyplain{}{\bfseries\leftmark}]{\fancyplain{}{\bfseries\thepage}}
\cfoot{}

%\pagestyle{headings}

\maketitle

\tableofcontents

\clearpage

%Chapter 1
\chapter{Introduction to GITM}

The Global Ionosphere Thermosphere Model (GITM) is a 3D model of the
upper atmosphere.  It runs for Earth, Mars and Titan.  A version is
being worked on for Jupiter.  GITM solves for the coupled continuity,
momentum and energy equations of the neutrals and ions.  For the ions,
the time rate of change of the velocity is ignored, so the
steady-state ion flow velocity is solved for.  The ion temperature is
a mixture of the electron and neutral temperature.

The neutrals are solved for using the Navier Stokes Equations.  The
continuity equation is solved for for each major species.  One of the
problems with GITM that needs to be rectified is that there are no
real tracer species, so a species is either solved for completely or
is not at all.  These species can still be included in the chemistry
calculation.  There is only one horizontal velocity that is computed,
while there are vertical velocities for each of the major species.  A
bulk vertical velocity is calculated as a mass weighted average.  The
temperature is a bulk temperature.

\section{Source Terms}

Chemistry is the only real source term for the continuity equation.
Typically, diffusion is added in the continuity equation to allow for
eddy diffusion, but this is not the case in GITM.  What happens is
that the vertical velocities are solved for, then a friction term is
applied to that the velocities stay very close together in the eddy
diffusion part of the code.  This way, the velocities can't differ too
much from each other.  Diffusion is not needed, then.

For the horizontal momentum equation, there are the following sources:
(1) ion drag; (2) viscosity; and (3) gravity wave acceleration.  For
the vertical velocity, the source terms are ion drag and friction
between the different neutral species.

For the neutral temperature, the following source terms are included:
(1) radiative cooling; (2) EUV heating; (3) auroral heating; (4) Joule
heating; (5) conduction; and (6) chemical heating.  The biggest pain
for the temperature equation is the use of a normalized temperature.
This means that the {\tt temperature} variable in GITM does not
contain the actual temperature, it contains the temperature multiplies
by Boltzmann's Constant divided by the mean mass.  This turns out to
be a factor that is very similar to the specific heat, or roughly or
order 1000.  In order to get the actual temperature, the variable has
to be multiplied by {\tt temp\_unit}.

\section{Ghost Cells}

GITM is a parallel code.  It uses a 2D domain decomposition, with the
altitude domain being the only thing that is not broken up.  Blocks of
latitude and longitude are used.  These blocks are then distributed
among different processors.  In order to communicate between the
processors, ghostcells are used.  These are cells that essentially
overlap with the neighboring block.  MPI (message passing interface)
is then used to move information from one block to another, filling in
the ghostcells.  The code then loops from 1-N, where the flux is
calculated at the boundaries from the 0-1 boundary to the N-N+1
boundary.  A second order scheme is used to calculate the fluxes,
along with a flux limiter.  Therefore, two ghost cells are needed.

In the vertical direction, ghost cells are also used to set boundary
conditions.  The values in these cells are used to calculate the
fluxes, just as described above.  Different types of boundary
conditions (constant values, constant fluxes, constant gradients,
floating, zero fluxes, etc) can be set by carefully choosing the right
values in the ghost cells.


\label{intro.ch}

%Chapter 2
\chapter{Chemistry}
\label{chemistry.ch}

\begin{equation}
O_2 \rightarrow 2O
\end{equation}
EUV dissociation rate.

\begin{equation}
N_2 \rightarrow 2N
\end{equation}
EUV dissociation rate.

\begin{equation}
O + O + M \rightarrow O_2 + M, \hspace{0.5 in}
R = 9.59 \times 10^{-46} e^{\frac{480}{T_n}}
\end{equation}

\begin{equation}
N_2 \rightarrow N_2^+
\end{equation}
EUV Ionization Rate.

\begin{equation}
N_2 \rightarrow N_2^+
\end{equation}
Auroral Ionization Rate.

\begin{equation}
O_2 \rightarrow O_2^+
\end{equation}
EUV Ionization Rate.

\begin{equation}
O_2 \rightarrow O_2^+
\end{equation}
Auroral Ionization Rate.

\begin{equation}
O(^3P) \rightarrow O^+(^4S)
\end{equation}
EUV Ionization Rate.

\begin{equation}
O(^3P) \rightarrow O^+(^2D)
\end{equation}
EUV Ionization Rate.

\begin{equation}
O(^3P) \rightarrow O^+(^2P)
\end{equation}
EUV Ionization Rate.

\begin{equation}
\begin{split}
O(^4S) \rightarrow & O^+(^4S) (40\%) \\
O(^4S) \rightarrow & O^+(^2D) (40\%) \\
O(^4S) \rightarrow & O^+(^2P) (20\%)
\end{split}
\end{equation}
Auroral Ionization Rate.

\begin{equation}
\begin{split}
O^+(^2D) + N_2 \rightarrow & N_2^+ + O + 1.33 eV \\
R = & 8.0\times 10^{-16}
\end{split}
\end{equation}

\begin{equation}
\begin{split}
O^+(^2P) + N_2 \rightarrow & N_2^+ + O + 3.02 eV \\
R = & 4.8\times 10^{-16}
\end{split}
\end{equation}

\begin{equation}
\begin{split}
N_2^+ + O_2 \rightarrow & O_2^+ + N_2 + 3.53 eV \\
R = & 5.0\times 10^{-17} \times \bigg(\frac{T_n+T_i}{600}\bigg)^{-0.8}
\end{split}
\end{equation}

\begin{equation}
\begin{split}
N_2^+ + O \rightarrow & NO^+ + N(^2D) + 0.7 eV \\
R = & 1.4\times 10^{-16} \times \bigg(\frac{T_n+T_i}{600}\bigg)^{-0.44}
\end{split}
\end{equation}

\begin{equation}
\begin{split}
N_2^+ + e^- \rightarrow & 2N(^2D) + 1.04 eV \\
R = & 1.8\times 10^{-13} \times \bigg(\frac{T_e}{300}\bigg)^{-0.39}
\end{split}
\end{equation}

\begin{equation}
\begin{split}
N_2^+ + O \rightarrow & O^+(^4S) + N_2 + 1.96 eV \\
R = & 1.4\times 10^{-16} \times \bigg(\frac{T_n+T_i}{600}\bigg)^{-0.44}
\end{split}
\end{equation}
I am not sure that this is correct, since it is the same as 2.8.

\begin{equation}
\begin{split}
N_2^2 + NO \rightarrow & NO^+ + N_2 + 6.33 eV \\
R = & 4.1 \times 10^{-16}
\end{split}
\end{equation}

\begin{equation}
\begin{split}
O^+(^4S) + O_2  \rightarrow & O_2^+ + O + 1.55 eV \\
R  =  & 2.82 \times 10^{-17} \\
      &    - 7.740\times 10^{-18}(T_{O2}/300.0) \\
      &    + 1.073\times 10^{-18}(T_{O2}/300.0)^2 \\
      &    - 5.170\times 10^{-20}(T_{O2}/300.0)^3 \\
      &    + 9.650\times 10^{-22}(T_{O2}/300.0)^4
\end{split}
\end{equation}

\begin{equation}
\begin{split}
O^+(^2D) + O_2 \rightarrow & O_2^+ + 4.865 eV \\
R = & 7.0 \times 10^{-16}
\end{split}
\end{equation}

\begin{equation}
\begin{split}
N^+ + O_2 \rightarrow & O_2^+ + N(^4S) + 2.486 \\
R = & 1.1 \times 10^{-16}
\end{split}
\end{equation}

\begin{equation}
\begin{split}
N^+ + O_2 \rightarrow & O_2^+ + N(^4D) + 0.1 \\
R = & 2.0 \times 10^{-16}
\end{split}
\end{equation}

\begin{equation}
\begin{split}
O_2^+ + e^- \rightarrow & O(^1D) + O(^1D) + 3.06 eV (31\%) \\
O_2^+ + e^- \rightarrow & O(^3P) + O(^1D) + 3.06 eV (42\%) \\
O_2^+ + e^- \rightarrow & O(^3P) + O(^3P) + 3.06 eV (22\%) \\
R = & 2.4\times 10^{-13} \bigg(\frac{T_e}{300}\bigg)^{-0.7}
\end{split}
\end{equation}

\begin{equation}
\begin{split}
O_2^+ + N(^4S) \rightarrow & NO^+ + O + 4.25 eV \\
R = & 1.5 \times 10^{-16}
\end{split}
\end{equation}

\begin{equation}
\begin{split}
O_2^+ + NO \rightarrow & NO^+ + O_2  + 2.813 eV \\
R = & 4.6 \times 10^{-16}
\end{split}
\end{equation}

\begin{equation}
\begin{split}
O_2^+ + N_2 \rightarrow & NO^+ + NO + 0.9333 eV \\
R = & 5.0 \times 10^{-22}
\end{split}
\end{equation}

\begin{equation}
\begin{split}
O^+(^2D) + O \rightarrow &  O^+(^4S) + O(^3P) + 3.31 eV \\
O^+(^2D) + O \rightarrow &  O^+(^4S) + O(^1D) + 1.35 eV \\
R = & 1.0 \times 10^{-17}
\end{split}
\end{equation}

\begin{equation}
\begin{split}
O^+(^2D) + e^- \rightarrow & O^+(^4S) + e^- + 3.31 eV  \\
R = & 7.8 \times 10^{-14}\bigg(\frac{T_e}{300}\bigg)^{-0.5}
\end{split}
\end{equation}

\begin{equation}
\begin{split}
O^+(^2D) + N_2 \rightarrow & O^+(^4S) + N_2 + 3.31 eV  \\
R = & 8.0 \times 10^{-16}
\end{split}
\end{equation}

\begin{equation}
\begin{split}
O^+(^2P) + O \rightarrow & O^+(^4S) + O + 5.0 eV  \\
R = & 5.2 \times 10^{-17}
\end{split}
\end{equation}

\begin{equation}
\begin{split}
O^+(^2P) + e^- \rightarrow & O^+(^4S) + e^- + 5.0 eV  \\
R = & 4.0 \times 10^{-14}\bigg(\frac{T_e}{300}\bigg)^{-0.5}
\end{split}
\end{equation}

\begin{equation}
\begin{split}
O^+(^2P) \rightarrow &  O^+(^4S) + 247.0nm\\
R = & 0.047
\end{split}
\end{equation}

\begin{equation}
\begin{split}
N^+ O_2 \rightarrow &  O^+(^4S) + NO + 2.31 eV\\
R = & 3.0 \times 10^{-17}
\end{split}
\end{equation}

\begin{equation}
\begin{split}
O^+(^4S) + N_2 \rightarrow &  NO^+ + N(^4S) + 1.10 eV\\
T_{eff} = & T_i + \frac{M_O}{M_O + M_{N_2}}\times\frac{M_{N_2} - M_b}{3k_b}V_i^2 \\
M_b = &\frac{M_c}{M_{mc}} \\
M_c = & \sum_n \frac{M_n\nu_{in}}{M_n + M_O} \\
M_{mc} = & \sum_n \frac{   \nu_{in}}{M_n + M_O} \\
R = & 1.533\times 10^{-18} - \\
    & 5.920\times 10^{-19} \bigg(\frac{T_{eff}}{300}\bigg) + \\
    & 8.600\times 10^{-20} \bigg(\frac{T_{eff}}{300}\bigg)^2 (T_{eff} < 1700) \\
R = & 2.730\times 10^{-18} - \\
    & 1.155\times 10^{-18} \bigg(\frac{T_{eff}}{300}\bigg) + \\
    & 1.483\times 10^{-19} \bigg(\frac{T_{eff}}{300}\bigg)^2 (T_{eff} > 1700)
\end{split}
\end{equation}
If $T_{eff} < 350$, then $T_{eff} = 350$.
If $T_{eff} > 6000$, then $T_{eff} = 6000$.

\begin{equation}
\begin{split}
O^+(^4S) + O_2 \rightarrow &  O_2^+ + O + 1.55 eV\\
R = & 2.820\times 10^{-17} - \\
    & 7.740\times 10^{-18} \bigg(\frac{T_{eff}}{300}\bigg) + \\
    & 1.073\times 10^{-18} \bigg(\frac{T_{eff}}{300}\bigg)^2 - \\ 
    & 5.170\times 10^{-20} \bigg(\frac{T_{eff}}{300}\bigg)^3 + \\ 
    & 9.650\times 10^{-22} \bigg(\frac{T_{eff}}{300}\bigg)^4\\
T_{eff} = & T_i + \frac{M_O}{M_O + M_{O_2}}\times\frac{M_{O_2} - M_b}{3k_b}V_i^2
\end{split}
\end{equation}
If $T_{eff} < 350$, then $T_{eff} = 350$.
If $T_{eff} > 6000$, then $T_{eff} = 6000$.

\begin{equation}
\begin{split}
O^+(^4S) + NO \rightarrow &  NO^+ + O + 4.36 eV\\
R = & 8.36\times 10^{-19} - \\
    & 2.02\times 10^{-19} \bigg(\frac{T_{eff}}{300}\bigg) + \\
    & 6.95\times 10^{-20} \bigg(\frac{T_{eff}}{300}\bigg)^2 (T_{eff} < 1500)\\
R = & 5.33\times 10^{-19} - \\
    & 1.64\times 10^{-20} \bigg(\frac{T_{eff}}{300}\bigg) + \\
    & 4.72\times 10^{-20} \bigg(\frac{T_{eff}}{300}\bigg)^2 \\
    & 7.05\times 10^{-22} \bigg(\frac{T_{eff}}{300}\bigg)^3 (T_{eff} > 1500) \\ 
T_{eff} = & T_i + \frac{M_O}{M_O + M_{NO}}\times\frac{M_{NO} - M_b}{3k_b}V_i^2
\end{split}
\end{equation}
If $T_{eff} < 350$, then $T_{eff} = 350$.
If $T_{eff} > 6000$, then $T_{eff} = 6000$.

\begin{equation}
\begin{split}
O^+(^4S) + N(^2D) \rightarrow &  N^+ + O + 1.45 eV\\
R = & 1.3 \times 10^{-16}
\end{split}
\end{equation}

\begin{equation}
\begin{split}
O^+(^2P) +e^- \rightarrow &  O^+(^2D) + e^- + 1.69 eV\\
R = & 1.3 \times 10^{-13} \bigg(\frac{T_e}{300}\bigg)^{-0.5} % te3m05
\end{split}
\end{equation}

\begin{equation}
\begin{split}
O^+(^2P) \rightarrow &  O^+(^2D) + 732nm\\
R = & 0.171
\end{split}
\end{equation}

\begin{equation}
\begin{split}
O^+(^2D) \rightarrow &  O^+(^4S) + 372.6nm\\
R = & 7.7\times 10^{-5}
\end{split}
\end{equation}

\begin{equation}
\begin{split}
O^+(^2P) + N_2 \rightarrow & N^+ + NO + 0.70 eV\\
R = & 1.0 \times 10^{-16}
\end{split}
\end{equation}

\begin{equation}
\begin{split}
O^+_2 + N(^2D) \rightarrow & N^+ + O_2 \\
R = & 2.5 \times 10^{-16}
\end{split}
\end{equation}

\begin{equation}
\begin{split}
O^+(^2P) + N \rightarrow & N^+ + O + 2.7 eV \\
R = & 1.0 \times 10^{-16}
\end{split}
\end{equation}

\begin{equation}
\begin{split}
O^+(^2D) + N \rightarrow & N^+ + O + 1.0 eV \\
R = & 7.5 \times 10^{-17}
\end{split}
\end{equation}

\begin{equation}
\begin{split}
N^+ + O_2 \rightarrow & NO^+ + O(^1D) + 6.67 eV \\
R = & 2.6 \times 10^{-16}
\end{split}
\end{equation}

\begin{equation}
\begin{split}
N^+ + O \rightarrow & O^+(^4S) + N + 0.93 eV \\
R = & 5.0 \times 10^{-19}
\end{split}
\end{equation}

\begin{equation}
\begin{split}
NO^+ + e^- \rightarrow & O + N(^2D) + 0.38 eV \\
R = & 4.0 \times 10^{-13} \bigg(\frac{T_e}{300}\bigg)^{-0.5} % te3m05
\end{split}
\end{equation}

\begin{equation}
\begin{split}
N(^2D) + e^- \rightarrow & N(^4S) + e^- + 2.38 eV\\
R = & 5.5 \times 10^{-16} \bigg(\frac{T_e}{300}\bigg)^{-0.5} % te3m05
\end{split}
\end{equation}

\begin{equation}
\begin{split}
N(^2D) + O \rightarrow & N(^4S) + O(^3P) + 2.38 eV (90\%)\\
N(^2D) + O \rightarrow & N(^4S) + O(^1D) + 0.42 eV (10\%)\\
R = & 2.0 \times 10^{-18}
\end{split}
\end{equation}

\begin{equation}
\begin{split}
N(^2D) \rightarrow & N(^4S) + 520nm \\
R = & 1.06 \times 10^{-5}
\end{split}
\end{equation}

\begin{equation}
\begin{split}
NO \rightarrow & N(^4S) + O \\
R = & 4.5 \times 10^{-6} e^{(-1\times 10^{-8}([O_2]\times 10^{-6})^{0.38})}
\end{split}
\end{equation}

\begin{equation}
\begin{split}
N(^4S) + O_2 \rightarrow & NO + O + 1.385 eV \\
R = & 4.4 \times 10^{-18} e^{-\frac{3220}{T_n}}
\end{split}
\end{equation}

\begin{equation}
\begin{split}
N(^4S) + NO \rightarrow & N_2 + O + 3.25 eV\\
R = & 1.5 \times 10^{-18} \sqrt{T_n}
\end{split}
\end{equation}

\begin{equation}
\begin{split}
N(^2P) \rightarrow & N(^2D) + 1040nm\\
R = & 7.9 \times 10^{-2}
\end{split}
\end{equation}

\begin{equation}
\begin{split}
N(^2D) + O_2 \rightarrow & NO + O(^3P) + 3.76 eV (90\%)\\
N(^2D) + O_2 \rightarrow & NO + O(^1D) + 1.80 eV (10\%)\\
R = & 6.2 \times 10^{-18} \frac{T_n}{300}
\end{split}
\end{equation}

\begin{equation}
\begin{split}
N(^2D) + NO \rightarrow & N_2 + O + 5.63 eV\\
R = & 7.0 \times 10^{-17}
\end{split}
\end{equation}

\begin{equation}
\begin{split}
O(^1D) \rightarrow & O(^3P) + 630nm\\
R = & 0.0071
\end{split}
\end{equation}

\begin{equation}
\begin{split}
O(^1D) \rightarrow & O(^3P) + 636.4nm\\
R = & 0.0022
\end{split}
\end{equation}
I don't understand this...


\begin{equation}
\begin{split}
O(^1D) + e^- \rightarrow & O(^3P) + e^- + 1.96 eV\\
R = & 2.6 \times 10^{-17} T_e^(0.5)e^(-22740/T_e) 
\end{split}
\end{equation}

\begin{equation}
\begin{split}
O(^1D) + N_2 \rightarrow & O(^3P) + N_2 + 1.96 eV\\
R = & 2.3 \times 10^{-17}
\end{split}
\end{equation}

\begin{equation}
\begin{split}
O(^1D) + O_2 \rightarrow & O(^3P) + O_2 + 1.96 eV\\
R = & 2.9 \times 10^{-17} e^{\frac{67.5}{T_n}}
\end{split}
\end{equation}

\begin{equation}
\begin{split}
O(^1D) + O(^3P) \rightarrow & 2O(^3P) + 1.96 eV\\
R = & 8.0 \times 10^{-18}
\end{split}
\end{equation}

\begin{equation}
\begin{split}
NO \rightarrow & NO^+ + e^-\\
R = & 5.88 \times 10^{-7}(1+0.2(F10.7-65)/100)e^F\cos(SZA)\\
F = & (\frac{[O_2]}{1\times 10^{6}})^{0.8855}
\end{split}
\end{equation}


%Bibliography
\clearpage
\addcontentsline{toc}{chapter}{Bibliography}
\bibliographystyle{chicagoa}
\bibliography{gitm}

\end{document}
